\documentclass{beamer}
\usetheme{CambridgeUS}

\setbeamertemplate{caption}[numbered]{}

\usepackage{enumitem}
\usepackage{tfrupee}
\usepackage{amsmath}
\usepackage{amssymb}
\usepackage{textcomp, gensymb}
\usepackage{graphicx}
\usepackage{txfonts}
                               
\providecommand{\pr}[1]{\ensuremath{\Pr\left(#1\right)}}
\providecommand{\mbf}{\mathbf}
\providecommand{\qfunc}[1]{\ensuremath{Q\left(#1\right)}}
\providecommand{\sbrak}[1]{\ensuremath{{}\left[#1\right]}}
\providecommand{\lsbrak}[1]{\ensuremath{{}\left[#1\right.}}
\providecommand{\rsbrak}[1]{\ensuremath{{}\left.#1\right]}}
\providecommand{\brak}[1]{\ensuremath{\left(#1\right)}}
\providecommand{\lbrak}[1]{\ensuremath{\left(#1\right.}}
\providecommand{\rbrak}[1]{\ensuremath{\left.#1\right)}}
\providecommand{\cbrak}[1]{\ensuremath{\left\{#1\right\}}}
\providecommand{\lcbrak}[1]{\ensuremath{\left\{#1\right.}}
\providecommand{\rcbrak}[1]{\ensuremath{\left.#1\right\}}}
\providecommand{\abs}[1]{\vert#1\vert}
\newcommand*{\permcomb}[4][0mu]{{{}^{#3}\mkern#1#2_{#4}}}
\newcommand*{\perm}[1][-3mu]{\permcomb[#1]{P}}
\newcommand*{\comb}[1][-1mu]{\permcomb[#1]{C}}

\newcounter{saveenumi}
\newcommand{\seti}{\setcounter{saveenumi}{\value{enumi}}}
\newcommand{\conti}{\setcounter{enumi}{\value{saveenumi}}}

\makeatletter
\newenvironment<>{proofs}[1][\proofname]{%
    \par
    \def\insertproofname{#1\@addpunct{.}}%
    \usebeamertemplate{proof begin}#2}
  {\usebeamertemplate{proof end}}
\makeatother

\title{Assignment 14}
\author{Kotikalapudi Karthik (CS21BTECH11030)}
\date{\today}
\logo{\large \LaTeX{}}


\begin{document}

% Title page frame
\begin{frame}
    \titlepage 
\end{frame}

% Remove logo from the next slides
\logo{}


% Outline frame
\begin{frame}{Outline}
    \tableofcontents
\end{frame}

%Question
\section{Question}
\begin{frame}{Question}
    \begin{block}{Probability, Random Variables and Stochastic Processes Chapter 8, Problem 8-31} 
         A die is tossed $102$ times, and the $i^{th}$ face shows $k_i$ = $18, 15, 19, 17, 13, \text{and } 20$ times. Test the hypothesis that the die is fair with $\alpha$ = 0.05 using the chi-square test
    \end{block}
\end{frame}

%Solution
\section{Solution}
\begin{frame}{Solution}
    Let's denote the random variable $X_1=\cbrak{1,2,3,4,5,6}$ where each $X_1=i$ denote that $i$ appeared on top of the die theoretically.\\
    Let's denote the random variable $X_2=\cbrak{1,2,3,4,5,6}$ where each $X_2=i$ denote that $i$ appeared on top of the die in the given case.
    \\
    Here no. of times die was thrown($n$) = $102$\\
    We know that the sum,
    \begin{align}
        \textbf{q} &= \sum^{6}_{i=1}\frac{\brak{n\pr{X_2=i}-n\pr{X_1=i}}^2}{n\pr{X_1=i}}
        \\
        \text{Here, }\pr{X_1=i} &= \frac{1}{6}, \forall i \in \cbrak{1,2,3,4,5,6}
        \\
        \implies \textbf{q} &= \sum^{6}_{i=1}\frac{\brak{\pr{X_2=i}-17}^2}{17}
        \\
        &= \frac{1+4+4+0+16+9}{17} = 2
    \end{align}
\end{frame}

\begin{frame}{Solution}
    If the die is fair,
    \begin{align}
        \textbf{q} &< {\chi}_{1-\alpha}^{2}(6-1)
        \\
        \implies \textbf{q} &< {\chi}_{0.95}^{2}(5)
        \\
        \text{The value of }{\chi}_{0.95}^{2}(5) &= 11.07
        \\
        \text{Clearly, }\textbf{q} &< 11.07
    \end{align}
    Therefore, we can accept that the die is fair. 
\end{frame}

\end{document}