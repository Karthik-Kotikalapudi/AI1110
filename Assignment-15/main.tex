\documentclass{beamer}
\usetheme{CambridgeUS}

\setbeamertemplate{caption}[numbered]{}

\usepackage{enumitem}
\usepackage{tfrupee}
\usepackage{amsmath}
\usepackage{amssymb}
\usepackage{textcomp, gensymb}
\usepackage{graphicx}
\usepackage{txfonts}

\def\inputGnumericTable{}

\usepackage[latin1]{inputenc}                                 
\usepackage{color}                                            
\usepackage{array}                                            
\usepackage{longtable}                                        
\usepackage{calc}                                             
\usepackage{multirow}                                         
\usepackage{hhline}                                           
\usepackage{ifthen}
\usepackage{caption} 
\providecommand{\mbf}{\mathbf}
\providecommand{\qfunc}[1]{\ensuremath{Q\left(#1\right)}}
\providecommand{\sbrak}[1]{\ensuremath{{}\left[#1\right]}}
\providecommand{\lsbrak}[1]{\ensuremath{{}\left[#1\right.}}
\providecommand{\rsbrak}[1]{\ensuremath{{}\left.#1\right]}}
\providecommand{\brak}[1]{\ensuremath{\left(#1\right)}}
\providecommand{\lbrak}[1]{\ensuremath{\left(#1\right.}}
\providecommand{\rbrak}[1]{\ensuremath{\left.#1\right)}}
\providecommand{\cbrak}[1]{\ensuremath{\left\{#1\right\}}}
\providecommand{\lcbrak}[1]{\ensuremath{\left\{#1\right.}}
\providecommand{\rcbrak}[1]{\ensuremath{\left.#1\right\}}}                                  
                               
\title{Assignment 15}
\author{Kotikalapudi Karthik (CS21BTECH11030)}
\date{\today}
\logo{\large \LaTeX{}}


\begin{document}
% Title page frame
\begin{frame}
    \titlepage 
\end{frame}

% Remove logo from the next slides
\logo{}


% Outline frame
\begin{frame}{Outline}
    \tableofcontents
\end{frame}
%Question
\section{Question}
\begin{frame}{Probability, Random Variables and Stochastic Processes Chapter 12, Problem 12-12}
    Show that if we use as estimate of the power spectrum $S\brak{\omega}$ of a discrete-time process $x[n]$ the function
    $$S_w\brak{\omega} = \sum_{m=-N}^{N}w_m R\sbrak{m}e^{-jm\omega T}$$
    then 
    $$S_w\brak{\omega}=\frac{1}{2\sigma}\int_{-\sigma}^{\sigma}S\brak{y}W\brak{\omega-y}dy$$ $$W\brak{\omega}=\sum^{N}_{-N}w_ne^{-jn\omega T}$$
\end{frame}

%Solution
\section{Solution}
\begin{frame}{Solution}
    Given,
    \begin{align}
        S_w\brak{\omega} = \sum_{m=-N}^{N}w_m R\sbrak{m}e^{-jm\omega T}
        \label{eq:eq-1}
    \end{align}
    We know that
    \begin{align}
        S\brak{y} &= \mathcal{F}\sbrak{R\sbrak{m}}
        \\
        \implies R\sbrak{m} &= \mathcal{F}^{-1}\sbrak{S\brak{y}}
        \\
        \implies R\sbrak{m} &= \frac{1}{2\sigma}\int^{\sigma}_{-\sigma}S\brak{y}e^{jmyT}dy
        \label{eq:eq-2}
    \end{align}
    Here $y$ is a variable in the frequency domain.\\
    If $T=1$sec, the limits will be $-\pi$ to $\pi$.\\
    As we don't know the value of $T$, let the limits be $-\sigma$ to $\sigma$.
\end{frame}
\begin{frame}{Solution}
    Substituting equation \eqref{eq:eq-2} in equation \eqref{eq:eq-1},
    \begin{align}
        S_w\brak{\omega} &= \sum_{m=-N}^{N}w_m \brak{\frac{1}{2\sigma}\int^{\sigma}_{-\sigma}S\brak{y}e^{jmyT}dy}e^{-jm\omega T}
        \\
        \implies S_w\brak{\omega} &= \frac{1}{2\sigma} \brak{\int^{\sigma}_{-\sigma}\brak{S\brak{y}\sum_{m=-N}^{N}w_me^{jmyT}}dy}e^{-jm\omega T}
        \\
        \implies S_w\brak{\omega} &= \frac{1}{2\sigma} \int^{\sigma}_{-\sigma}\brak{S\brak{y}\sum_{m=-N}^{N}w_me^{-jm(\omega-y)T}}dy
        \label{eq:eq-3}
    \end{align}
    
\end{frame}

\begin{frame}{Solution}
    $W\brak{w}$ is the Discrete-time Fourier Transform(DTFT) of $w_n$
    \begin{align}
        W\brak{\omega} &=\sum_{n=-N}^{N}w_ne^{-jm\omega T}
        \\
        \implies W\brak{\omega-y} &=\sum_{n=-N}^{N}w_ne^{-jm(\omega-y)T}
        \label{eq:eq-4}
    \end{align}
    Substituting equation \eqref{eq:eq-4} in equation \eqref{eq:eq-3},
    \begin{align}
        S_w\brak{\omega} &= \frac{1}{2\sigma} \int^{\sigma}_{-\sigma}\brak{S\brak{y}W(\omega-y)}dy
    \end{align}
\end{frame}

\end{document}
